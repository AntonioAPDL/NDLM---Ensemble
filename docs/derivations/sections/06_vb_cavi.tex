\section{Mean-Field Variational Bayes (CAVI)}

\subsection{Factorization}
Use
\begin{align}
q(\text{states},\sigma^2,W,\lambda)
= q(\text{states})\prod_{j=0}^J q(\sigma_j^2)\prod_b q(\bm W_b)q(\lambda).
\label{eq:vb_factorization}
\end{align}
Each coordinate update uses the generic CAVI rule
\begin{align}
\log q_i^\star(\vartheta_i)
= \E_{q_{-i}}[\log p(\text{data},\text{all unknowns})] + \text{const},
\label{eq:vb_cavi_rule}
\end{align}
then identifies the corresponding exponential-family kernel.

\subsection{State Factor}
Given current expected precisions,
\begin{align}
q(\text{states})\propto
\exp\Big(\E_{-\text{states}}[\log p(\text{data},\text{states},\sigma^2,W,\lambda)]\Big),
\end{align}
which remains Gaussian and is computed by a Kalman smoother on a pseudo-model with
$\E_q[1/\sigma_j^2]$ and $\E_q[\bm W_b^{-1}]$ inserted.
This is the direct Gaussian kernel implied by \eqref{eq:vb_cavi_rule} and
\eqref{eq:log_joint_A}--\eqref{eq:log_joint_C}.

\subsection{Variance Factors}
For each $j$,
\begin{align}
q(\sigma_j^2) &= \IG(\tilde a_{\sigma j},\tilde b_{\sigma j}), \\
\tilde a_{\sigma j} &= a_{\sigma j}+\frac{N_j}{2}, \\
\tilde b_{\sigma j} &= b_{\sigma j}+\frac{1}{2}\E_q[\mathrm{SSE}_j].
\label{eq:vb_sigma}
\end{align}
These parameters come from collecting terms in
$\log \sigma_j^2$ and $\sigma_j^{-2}$ under \eqref{eq:vb_cavi_rule}.

\subsection{Evolution Covariance Factors}
For each block $b$,
\begin{align}
q(\bm W_b) &= \IW(\tilde\nu_b,\tilde{\bm S}_b), \\
\tilde\nu_b &= \nu_b + T_b, \\
\tilde{\bm S}_b &= \bm S_b + \sum_{t=1}^{T_b} \E_q[\bm e_t\bm e_t^\T].
\label{eq:vb_W}
\end{align}
Again, \eqref{eq:vb_cavi_rule} yields an inverse-Wishart kernel by grouping
$\log|\bm W_b|$ and $\tr(\bm W_b^{-1}\cdot)$ terms.

\subsection{Transfer Coefficient Factor}
With Gaussian prior and constant $w^\zeta$,
\begin{align}
q(\lambda)=\N(\tilde m_\lambda,\tilde C_\lambda),
\end{align}
where
\begin{align}
\tilde C_\lambda^{-1}
&= C_{\lambda0}^{-1}+\frac{1}{w^\zeta}\sum_{t=1}^T\E_q[\zeta_{t-1}^2], \\
\tilde m_\lambda
&= \tilde C_\lambda\left(C_{\lambda0}^{-1}m_{\lambda0}+\frac{1}{w^\zeta}\sum_{t=1}^T\E_q[\zeta_{t-1}(\zeta_t-\bm x_t^\T\bm\psi_t)]\right).
\label{eq:vb_lambda}
\end{align}
This is the Normal kernel from the quadratic form in $\lambda$ induced by
\eqref{eq:vb_cavi_rule} and the Gaussian transition model.

\subsection{Expectations Required Each Iteration}
\begin{itemize}
\item $\E_q[1/\sigma_j^2]$, $\E_q[\log\sigma_j^2]$,
\item $\E_q[\bm W_b^{-1}]$, $\E_q[\log|\bm W_b|]$,
\item $\E_q[\bm x_t]$, $\E_q[\bm x_t\bm x_t^\T]$, $\E_q[\bm x_t\bm x_{t-1}^\T]$,
\item $\E_q[\mathrm{SSE}_j]$ and $\E_q[\bm e_t\bm e_t^\T]$.
\end{itemize}
For ragged-horizon Model C, these expectations are computed on lead-specific active
states (or equivalent embedded fixed-dimension states) with $n_k$ observations at lead $k$.

Closed forms used in computation are:
\begin{align}
\E_q[\sigma_j^{-2}] &= \frac{\tilde a_{\sigma j}}{\tilde b_{\sigma j}},
& \E_q[\log \sigma_j^2] &= \log \tilde b_{\sigma j}-\psi(\tilde a_{\sigma j}), \label{eq:vb_sigma_moments}\\
\E_q[\bm W_b^{-1}] &= \tilde\nu_b\,\tilde{\bm S}_b^{-1},
& \E_q[\log|\bm W_b|] &= \log|\tilde{\bm S}_b|-\sum_{i=1}^{d_b}\psi\!\Big(\frac{\tilde\nu_b+1-i}{2}\Big)-d_b\log 2, \label{eq:vb_iw_moments}\\
\E_q[\bm x_t\bm x_t^\T] &= \bm C_t^\star+\bm m_t^\star(\bm m_t^\star)^\T,
& \E_q[\bm x_t\bm x_{t-1}^\T] &= \bm C_{t,t-1}^\star+\bm m_t^\star(\bm m_{t-1}^\star)^\T. \label{eq:vb_state_moments}
\end{align}
The cross-covariance $\bm C_{t,t-1}^\star$ is obtained from the backward smoother recursion
used to compute the Gaussian state factor.
